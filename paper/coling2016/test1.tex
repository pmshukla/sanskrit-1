% This is a Unicode file.
\documentclass[12pt]{article}
\usepackage{multicol} % just to get narrow columns on one page
\usepackage{polyglossia} % the multilingual support package
% for XeLaTeX - includes Sanskrit.
% Next, from the polyglossia manual:
\setdefaultlanguage{english} % this is mostly going to be English text, with
\setotherlanguage{sanskrit} % some Sanskrit embedded in it.
% These will call appropriate hyphenation.
\usepackage{xltxtra} % standard for nearly all XeLaTeX documents
\defaultfontfeatures{Mapping=tex-text} % ditto
\setmainfont{Gandhari Unicode} %could be any Unicode font
% Now define some Devanagari fonts:
% John Smith's Nakula, input using Velthuis transliteration
\newfontinstance
\velthuis [Script=Devanagari,Mapping=velthuis-sanskrit]{Nakula}
% John Smith's Sahadeva, input using Velthuis transliteration
\newfontinstance
\sahadeva [Script=Devanagari,Mapping=velthuis-sanskrit]{Sahadeva}
% John's Sahadeva, input using scholarly romanisation in Unicode
\newfontinstance
\sahadevaunicode [Script=Devanagari,Mapping=RomDev]{Sahadeva}
% Microsoft's Mangal font (ugh!), input using standard romanisation in Unicode.
\newfontinstance
\mangal [Script=Devanagari,Mapping=RomDev]{Mangal}
% Somdev's RomDev.map is used above to get the mapping
% from Unicode -> Devanāgarī. Zdenek Wagner's velthuis-sanskrit.map
% is used to get the Velthuis->Devanāgarī mapping. These are the files
% that XeTeX uses to make all the conjunct consonants without needing
% any external preprocessor (like the old devnag.c program).
% % Set up the font commands:
%
\newcommand{\velt}[1]{{\velthuis \textsanskrit{#1}}}
\newcommand{\saha}[1]{{\sahadeva\textsanskrit{#1}}}
\newcommand{\sahauni}[1]{{\sahadevaunicode\textsanskrit{#1}}}
\newcommand{\mangaluni}[1]{{\mangal\textsanskrit{#1}}}
% \textssanskrit, above, is a Polyglossia command that gets Sanskrit hyphenation right.
% ... and here we go!
\begin{document}
\begin{multicols}{2} % narrow cols to force plenty of hyphenation
\large % ditto.
\begin{enumerate}
\item[1]
With Xe\LaTeX\ it's easy to typeset Velthuis encoded Devanagari like the following example, without using a preprocessor:
\velt{sugataan sasutaan sadharmakaayaan pra.nipatyaadarato 'khilaa.m"sca vandyaan|
sugataatmajasa.mvaraavataara.m kathayi.syaami yathaagama.m samaasaat||} Bodhicaryāvatāra 1,1.
NB: automatic hyphenation.
\item[2]
A different Devanāgarī font:
\saha{sugataan sasutaan sadharmakaayaan pra.nipatyaadarato 'khilaa.m"sca vandyaan|
sugataatmajasa.mvaraavataara.m kathayi.syaami yathaagama.m samaasaat||} Bodhicaryāvatāra 1,1.
\item[3]
Another sentence: \saha{ratnojjvalastambhamanorame.su muktaamayodbhaasivitaanake.su|
svacchojjvalasphaa.tikaku.t.time.su sungandhi.su snaanag.rhe.su te.su||} 2,10.
\item[4]
Now, thanks to Somdev's RomDev.map, we can input in Unicode, using standard scholarly transliteration, and get Devanāgarī generated for us automatically:
\sahauni{āsīdrājā nalo nāma vīrsenasuto balī||\par }
\item[5]
Plain Unicode input, no tricks:
āsīdrājā nalo nāma vīrsenasuto balī||
\item[6]
Plain Unicode romanisation input, no tricks:
\mangaluni{āsīdrājā nalo nāma vīrsenasuto balī||\par }
Plain Unicode Devanāgarī input, no tricks:
{\mangal आसीद्राजा नलो नाम वीरसेनसुतो बली|\par}
\end{enumerate}
\end{multicols}

\noindent
English and Devanāgarī are both doing okay. The only thing that isn't hyphenating well yet is Sanskrit in roman transliteration.

Other nice stuff becomes easy. E.g., define a command \verb|\example| that prints a romanised word in Nāgarī, and then repeats it in romanisation, in parentheses:

\verb|\newcommand\example[1]{\sahauni{#1}~(\emph{#1})}|\newcommand\example[1]{\sahauni{#1}~(\emph{#1})}

Input: \verb|\example{ekadhā}|

Output: \example{ekadhā}
\end{document}
%that's all folk