\documentclass[12pt]{article}
\usepackage{tipa}
\usepackage{hyperref}
\hypersetup{
%    bookmarks=true,         % show bookmarks bar?
%    unicode=false,          % non-Latin characters in Acrobat�s bookmarks
%    pdftoolbar=true,        % show Acrobat�s toolbar?
%    pdfmenubar=true,        % show Acrobat�s menu?
%    pdffitwindow=false,     % window fit to page when opened
    pdfstartview={FitH},    % fits the width of the page to the window
    pdftitle={Pronunciation of Nityanand Misra},    % title
    pdfauthor={Nityanand Misra},     % author
%    pdfsubject={Subject},   % subject of the document
%    pdfcreator={Creator},   % creator of the document
%    pdfproducer={Producer}, % producer of the document
%    pdfkeywords={keyword1} {key2} {key3}, % list of keywords
%    pdfnewwindow=true,      % links in new window
    colorlinks=true,       % false: boxed links; true: colored links
    linkcolor=red,          % color of internal links
    citecolor=green,        % color of links to bibliography
    filecolor=magenta,      % color of file links
    urlcolor=cyan           % color of external links
}


\title{Pronunciation of my name}
\author{Nity\={a}nanda Mi\'sra}
\date{December 22, 2010}

\begin{document}
\maketitle						% automatic title!

\section{Why this exercise}

Many people unfamilar with \href{http://en.wikipedia.org/wiki/Sanskrit}{Sanskrit}, the language of my name, have some trouble pronouncing the name. Most Indians are exposed to a great many Sanskrit names and words since childhood, and usually get the pronunciation right. This document is primarily intended for people who have relatively less or no exposure to Sanskrit words and want to pronounce my name correctly.\\

The example sounds referred to in this document are from European languages. While Sanskrit and European languages do belong to the same family of \href{http://en.wikipedia.org/wiki/Indo-European_languages}{Indo-European languages}, the real reason for including examples from European languages only is that the Latin script, which I am comfortable with, can be use to write most of them.

\section{Plain and simple for the phoneticians}

\subsection{Remember Henry Higgins in \emph{My Fair Lady}?}

Using the \href{http://en.wikipedia.org/wiki/International_Phonetic_Alphabet}{International Phonetic Alphabet} (IPA) notation, the pronunciation of my name is \textipa{\|[ni\|[tjA:\|[n5\|[n\|[d5} \textipa{miC\*r5}. That's it!

\subsection{What's \emph{Nity\={a}nanda Mi\'sra} then?}

That is the version as per the \href{http://en.wikipedia.org/wiki/International_Alphabet_of_Sanskrit_Transliteration}{International Alphabet for Sanskrit Transliteration} (IAST), a popular transliteration scheme for Sanskrit.

\subsection{And how about \emph{Nityanand Misra}?}

Latin script for English has only 26 alphabets, so the name is spelt without the IAST diacritical marks. I do not spell the terminal `a' in my first name when written using the English alphabet. This is due to a phonetic practice known as \href{http://en.wikipedia.org/wiki/Schwa#Schwa_syncope_in_Hindi}{Schwa syncope}, prevalent in my mother tongue \href{http://en.wikipedia.org/wiki/Hindi}{Hindi} and some other Indian languages. Many Sanskrit words ending with the near-open central vowel are pronounced without the terminal vowel when used in vernacular Indian languages (as opposed to when being spoken in Sanskrit). When such Sanskrit words are spelt in English in India, the terminal `a' is often omitted. Ideally I should include the `a' at the end, but it would only add to confusion in India. Plus, it is too late to change it everywhere!

\section{For the phonetically uninitiated}

\subsection{My first name}

My first name consists of four syllables. These are described below. \\

\textbf{nit} (IPA \textipa{\|[ni\|[t}) - This is the dental nasal consonant, followed by the close front unrounded vowel and then the voiceless dental plosive consonant. The first consonant is the sound in Finnish \emph{ka\textbf{n}to}, French \emph{co\textbf{nn}exion}, Polish \emph{\textbf{n}oga}, Portuguese \emph{\textbf{n}ariz}, Spanish \emph{a\textbf{n}tes} and Swedish \emph{\textbf{n}od}. The vowel is the sound in Danish \emph{b\textbf{i}l\textbf{i}st}, Dutch \emph{b\textbf{ie}t}, English \emph{\textbf{E}ngland}, French \emph{f\textbf{i}n\textbf{i}}, Italian \emph{b\textbf{i}le} and Spanish \emph{t\textbf{i}po}. The last consonant is the sound in Finnish \emph{\textbf{t}u\textbf{tt}i}, Italian \emph{\textbf{t}ale}, Polish \emph{\textbf{t}om}, Portuguese \emph{mon\textbf{t}anha}, and Spanish \emph{\textbf{t}ango}. \\

\textbf{y\={a}} (IPA \textipa{jA:}) - This is the palatal approximant followed by the open back unrounded vowel. The consonant sound is seen in Czech \emph{\textbf{j}e}, Dutch \emph{\textbf{j}aar}, English \emph{\textbf{y}ou}, French \emph{\textbf{y}aourt}, German \emph{\textbf{J}och}, Norwegian \emph{\textbf{j}ul} and Swedish \emph{\textbf{j}ag}. The vowel sound is the same as the one in Ducth \emph{b\textbf{a}d}, English \emph{sp\textbf{a}}, German \emph{T\textbf{a}g} and Norwegian (also Swedish) \emph{h\textbf{a}t}. \\

\textbf{nan} (IPA \textipa{\|[n5\|[n}) - The two consonants are the same as the first consonant in the first syllable (dental nasals). The vowel sound in between is the near-open central vowel, and is the sound heard in Danish \emph{spis\textbf{er}}, English \emph{n\textbf{u}t}, German \emph{Ob\textbf{er}} and Portuguese \emph{sac\textbf{a}}. \\

\textbf{da} (IPA \textipa{\|[d5}) - This is the voiced dental plosive followed by the near-open central vowel. The vowel sound is the same as in the third syllable, while the consonant is the sound in English \emph{\textbf{th}at}, Italian \emph{\textbf{d}are}, Polish \emph{\textbf{d}om}, Portuguese \emph{\textbf{d}ar}, Spanish \emph{hun\textbf{d}ido} and Swedish \emph{\textbf{d}ag}. \\

\subsection{My last name}

My last name consists of two syllables. These are described below. \\

\textbf{mi\'s} (IPA \textipa{miC}) - This is the bilabial nasal consonant followed by the close front unrounded vowel and the voiceless alveolo-palatal fricative. The first consonant and the following vowel together sound like the first two letters in the Enlish word \emph{\textbf{mi}neral}. The second consonant is similar to the sound in the English word \emph{\textbf{sh}ip}, but is more precisely the sound in Danish \emph{\textbf{sj}\protect{\ae}l}, Polish \emph{\textbf{\'s}ruba} or Swedish \emph{\textbf{kj}ol}.  \\

\textbf{ra} (IPA \textipa{\*r5}) - This is the alveolar approximant followed by the oear-open central vowel. The consonant sound is present in the English word \emph{\textbf{r}ed}. Note that this is not the alveolar trill which is also known as the \emph{rolled r}. The vowel sound has been explained previously. \\

\section{A brief on the meaning}

In this section I follow the IAST notation. 

\subsection{Meaning of \emph{Nity\={a}nanda}}

\emph{Nitya} means eternal in Sanskrit while \emph{\={A}nanda} means bliss. The \href{http://en.wikipedia.org/wiki/Sandhi}{morphophonological} combination of the two results in \emph{Nity\={a}nanda}. The literal translation is then `eternal bliss', but the compound is to be taken as \href{http://en.wikipedia.org/wiki/Bahuvrihi}{exocentric}. In this context, the meaning is `he who is eternally blissful', or `he who finds bliss in what is eternal'. 

\subsection{Meaning of \emph{Mi\'sra}}

\emph{Mi\'sra} in Sanskrit means blended or combined, and secondarily diverse or manifold. The \href{http://en.wikipedia.org/wiki/Vedas}{\emph{Veda}s}, the oldest scriptures of Hinduism, have primarily two parts -- one focussing on \emph{Karma} and rituals, and the other on philosophy and knowledge. The people who started blending the teachings of the two parts (as a sort of middle path), came to be known as \emph{Mi\'sra}s.

\end{document}             % End of document 

